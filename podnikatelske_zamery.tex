\documentclass[12pt, a4paper]{article}

\usepackage[T1]{fontenc}
\usepackage[utf8]{inputenc}

\usepackage{blindtext}
\usepackage{titlesec}
\usepackage{array}
\usepackage{url}
\usepackage[czech]{babel}
\usepackage[backend=biber]{biblatex}
\usepackage{csquotes}
\usepackage{amsmath}
\usepackage[bottom]{footmisc}
\usepackage{caption}
\usepackage{tabularx}
\addbibresource{./zdroje.bib}

\usepackage{graphicx}

\graphicspath{files/}

\usepackage{tocloft}
\renewcommand{\cftsecleader}{\cftdotfill{\cftdotsep}}

\begin{document}

\nocite{csob-podnikatelsky-plan}
\nocite{pz-idoklad}
\author{Jan Prokůpek, Miroslav Sklenář}
\title{\textbf{Podnikatelský záměr}\break Hats for Cats s.r.o.}
\date{}

\maketitle

\pagebreak

\tableofcontents

\pagebreak

\section{Shrnutí}

\subsection{Jméno a místo podnikání}
Jedná se o fiktivní firmu \textbf{Hats for Cats s.r.o.} se sídlem v Brně.

\subsection{Obchodní koncept}
Hats for Cats je originální značka specializující se na výrobu a prodej stylových, pohodlných a funkčních čepic pro kočky. 
Naším cílem je přinést do světa mazlíčků špetku originality a elegance, která potěší jak majitele, tak jejich kočičí společníky.

\subsection{Prodávaný produkt/služba}
Nabízíme ručně vyráběné čepice pro kočky v různých stylech a velikostech, 
které jsou přizpůsobeny pohodlí a bezpečnosti zvířete. 
Produkty zahrnují sezónní kolekce (zimní čepice, letní kloboučky) i tematické modely (vánoční, narozeninové, apod.),
či čepice kompletně na míru.

\subsection{Plněná potřeba na trhu}
V současnosti chybí na trhu esteticky atraktivní a funkční doplňky pro kočky, které by splňovaly požadavky majitelů na kvalitu a design. Hats for Cats vyplňuje tuto mezeru a oslovuje rostoucí komunitu milovníků koček, kteří hledají jedinečné produkty pro své mazlíčky.

\subsection{Konkurenční výhoda}
Naše konkurenční výhody spočívají v unikátním designu, použití kvalitních a hypoalergenních materiálů a důrazu na ruční výrobu. Dále se zaměřujeme na silný marketing na sociálních sítích, který osloví cílovou skupinu mladých a kreativních zákazníků.
Mezi existující alternativy patří levné a nekvalitní produkty z Číny, které nejsou přizpůsobeny potřebám zvířat a často mohou
způsobit zdravotní problémy kočkám, nebo tržiště jako Etsy.

\subsection{Ziskovost}
Očekáváme, že během prvního roku dosáhneme návratnosti investic díky 
nízkým provozním nákladům 
a atraktivní marži na našich produktech.

\subsection{Momentální situace}
Projekt je ve fázi příprav, zahrnující tvorbu prvních prototypů, 
testování na trhu a sestavování marketingové kampaně.

\subsection{Účel podnikatelského plánu}
Podnikatelský plán slouží k získání investice na zahájení výroby, 
marketingovou propagaci a detekce distribuční sítě.

\subsection{Potřeba investice}
Základní vklad na zahájení výroby, marketing, materiály,
pronájem prostor a další, bude činit cca 530 000 Kč.
Z této částky je 250 000 Kč předem uhrazeno zakladateli.

\section{Představení společnosti}
Společnost \textbf{Hats for Cats s.r.o.} byla založena 2 fyzickými osobami,
které podnikají dle živnostenského oprávnění. Dle společenské smlouvy
každý z majitelů vlastní 50\% podíl na společnosti a má rovnocenný podíl v zisku.
Jednatelem společnosti je Jan Prokůpek.

\vspace{10pt}

\noindent Hats for Cats s.r.o. je společnost zaměřená na výrobu a prodej různých typů čepic pro kočky.
Nabízí ručně vyráběné produkty, které jsou uzpůsobeny několika velikostem, tak aby seděly na většinu koček.
Lze si také nechat vyrobit čepici na míru.

\pagebreak

\section{Podnikatelský projekt}

Hats for Cats se zaměřuje na výrobu, prodej a distribuci módních doplňků pro domácí mazlíčky, konkrétně kočky. 
Hlavní obory podnikání zahrnují:

\begin{enumerate}
  \item Výroba textilních doplňků. Jedná se o ekonomickou činost s kódem CZ-NACE 14.19.
  \item Maloobchod a e-commerce - kód CZ-NACE 47.91.
\end{enumerate}

\subsection{Vstupní předpoklady}

\subsubsection{Oprávnění k provozování podniku}
\begin{itemize}
  \item Získání potřebného kapitálu.
  \item Založení společnosti s ručením omezeným (s.r.o.):
  \begin{itemize}
    \item Zápis do obchodního rejstříku.
    \item Zaplacení správního poplatku.
    \item Získání IČO.
    \item Notářsky ověřený zápis společenské smlouvy mezi společníky.
  \end{itemize}
  \item{Registrace k daním} - pokud bude obrat vyšší než 2 mio. Kč, bude nutné se registrovat k dani z přidané hodnoty.
\end{itemize}

\subsubsection{Požadavky na zaměstnance}
Zaměstnanci by měli být schopni ručně šít a zpracovávat textilní materiály.

\subsubsection{Materiální a technické vybavení}
\begin{itemize}
  \item Prostory pro výrobu, skladování a administrativu.
  \item Šicí stroje, nůžky, střihové šablony, materiály a ostatní vybavení pro výrobu.
  \item Zázemí pro administrativu a marketing(počítač, tiskárna atd.).
\end{itemize}

\subsection{Organizačně-právní forma podnikání}
Podnik bude již od počátku fungovat jako společnost s ručením omezeným (s.r.o.).
Tato forma podnikání byla zvolena pro možnost oddělení osobního majetku majitelů od majetku společnosti.
Zároveň umožňuje získání investic od externích investorů.

\subsection{Stadium rozvoje podniku}
Společnost se nacházi ve fázi zahájení(start-up). Zde bude kladen důraz primárně na:
\begin{itemize}
  \item Budování značky a získání prvních zákazníků
  \item Nastavení výroby a dodavatelských řetězců
  \item Vytvoření e-shopu a marketingové strategie
\end{itemize}

\subsection{Majetkoprávní vztahy}

\begin{itemize}
  \item Hmotný majetek
  \begin{itemize}
    \item Prostory pro výrobu, skladování a administrativu.
    \item Šicí stroje, nůžky, střihové šablony, materiály a ostatní vybavení pro výrobu.
    \item Zázemí pro administrativu a marketing(počítač, tiskárna atd.).
  \end{itemize}
  \item Nehmotný majetek
  \begin{itemize}
    \item Značka Hats for Cats
    \item Webové stránky(doména, hosting), účty na sociálních sítích
    \item Know-how a zkušenosti zakladatelů
    \item Grafické návrhy produktů, design a marketingové materiály.
  \end{itemize}
  \item{Finanční majetek} - základní kapitál společnosti, zdroje na zahájení výroby, popř. jinak zprostředkované zdroje.
\end{itemize}

\pagebreak

\subsection{Organizace podniku}

Hats for Cats s.r.o. bude mít systém řízení, který zahrnuje různé úrovně odpovědnosti.
Na vrcholové úrovni bude jednatel společnosti, který bude zodpovědný za celkové řízení a strategii firmy.
Pod ním bude vedoucí výroby, který bude mít na starosti koordinaci výroby čepic a řízení zaměstnanců v této oblasti.
Vedoucí bude ovšem i normálním zaměstnancem, který se bude podílet na výrobě.
Administrativu a marketing bude zajišťovat jednatel společně se druhým zakladatelem společnosti.
Budou se starat o komunikaci s dodavateli, zákazníky, správu objednávek a marketingové kampaně.

\vspace{10pt}

\begin{figure}[h]
  \centering
  \includegraphics[width=0.8\textwidth]{files/obr1.png}
  \caption{Organizační struktura společnosti}
\end{figure}

Sídlo společnosti se nachází v Brně, konkrétně v kancelářské budově v blízkosti centra města, což usnadňuje přístup pro zaměstnance i zákazníky.
Prostor je vybavený moderními kancelářemi a výrobními dílnami pro výrobu čepic.
Orientační bod je blízkost hlavního nádraží, což usnadňuje dopravu do firmy.
K dispozici je parkování pro zaměstnance i návštěvníky, a to v areálu budovy, což zajišťuje pohodlný přístup.
\vspace{10pt}

Firma bude fungovat s malým administrativním zázemím pro správu objednávek, komunikaci s dodavateli a zákazníky.
Provozní doba firmy bude od pondělí do pátku, od 8:00 do 16:00 hodin.
V této době bude probíhat jak výroba, tak i administrativa.
Firma se zaměří na efektivní komunikaci a rychlé dodání produktů zákazníkům.

\pagebreak

\subsubsection{BOZP}

Naše firma se zaměřuje na ruční výrobu čepic pro kočky. 
Vzhledem k povaze práce je nutné zajistit bezpečné pracovní prostředí pro všechny zaměstnance. 
Hlavní rizika se týkají práce s textilními materiály, šicími stroji, ostrými nástroji a ergonomie při dlouhodobém sezení.
\vspace{10pt}

Následující rizika by měla být zohledněna v rámci BOZP(tento list není konečný, avšak obsahuje rizika s nejvyšším potenciálem poškození):

\begin{enumerate}
  \item Riziko poranění ostrými nástroji - nůžky, jehly, špendlíky.
  \item Riziko poranění šicími stroji - šití na strojích s rizikem šití prstů.
  \item Riziko úrazu elektrickým proudem - při používání elektrických strojů.
  \item Riziko vzniku požáru - při práci s hořlavými materiály.
\end{enumerate}

\vspace{10pt}

Pro předejití těmto rizikům je nutné zajistit pravidla BOZP, která budou dodržována všemi zaměstnanci.
Je nutné všechny zaměstnance seznámit s těmito pravidly a pravidelně je kontrolovat. Tento list
obsahuje základní opatření, která by měla být dodržována:

\begin{enumerate}
  \item Používání ochranných pomůcek - rukavice, brýle, roušky.
  \item Pravidelná údržba strojů a kontrola jejich bezpečnosti.
  \item Pravidelná kontrola elektrických zařízení.
  \item Důrazné proškolení zaměstnanců pro práci s elektrickými zařízeními.
  \item Umístění hasicích přístrojů a školení zaměstnanců v jejich používání.
  \item Zákaz používání otevřeného ohně(včetně kouření) v dílně.
\end{enumerate}

\pagebreak

\subsection{Popis vyráběného výrobku}

Hats for Cats se zaměřuje na výrobu čepic pro kočky. Čepice jsou vyráběny 
z kvalitních a hypoalergenních materiálů, které jsou přizpůsobeny pohodlí a bezpečnosti zvířete.
Produkty zahrnují sezónní kolekce (zimní čepice, letní kloboučky) i tematické modely (vánoční, narozeninové, apod.),
či čepice kompletně na míru. Lze se je objednat v různých velikostech, tak aby seděly na většinu koček.

\begin{figure}[h!]
  \centering
  \includegraphics[width=0.35\textwidth, height=0.3\textwidth]{files/kocka1.png}
  \hspace{0.5cm}
  \vspace{0.5cm}
  \includegraphics[width=0.35\textwidth, height=0.3\textwidth]{files/kocka2.png}
  \hspace{0.5cm}
  \includegraphics[width=0.35\textwidth, height=0.3\textwidth]{files/kocka3.png}
  \caption{Obrázky koček s čepicemi}
\end{figure}

\section{Realizace}
\subsection{Výroba}

Jak již bylo zmíňeno, výroba se odehrává v prostorách firmy v Brně. 
\vspace{10pt}

\noindent Výroba čepic začíná výběrem kvalitních materiálů, které jsou jemné, elastické a bezpečné pro kočičí pokožku.
Používáme především přírodní látky jako bavlnu a vlnu, které jsou prodyšné a hypoalergenní.
Každý kus je ručně střižen, šit a dekorován našimi zkušenými zaměstnanci, kteří mají dlouholeté zkušenosti s textilní výrobou.

Výrobní proces obecně zahrnuje následující kroky:

\begin{enumerate}
  \item Výběr materiálů: Vybíráme látky, které jsou měkké a šetrné k pokožce koček.
  \item Střih a příprava: Látky jsou ručně stříhány podle předem navržených vzorců.
  \item Šití: Čepice jsou šity na šicích strojích
  \item Dekorace a dokončení: Pokud má čepice dekorace, jsou přidány ručně.
\end{enumerate}

Mezi materiály, které používáme, patří bavlna, jemná vlna a elastické tkaniny, které umožňují snadné nasazení a pohodlné nošení.

\subsubsection{Výrobní kapacita}

Výrobní kapacita firmy je závislá na počtu zaměstnanců, dostupných prostorách a poptávce.
Pro začátek se můžeme řídit následující rovnicí pro přibližný výpočet výrobní kapacity za měsíc:

\begin{equation}
  \text{Výrobní kapacita} = \text{zaměstnanci} \cdot \text{čepic za den} \cdot 20
\end{equation}

Pokud bysme měli 3 zaměstnance(1 vedoucí výroby a 2 šičky) a každá švadlena by mohla vyrobit asi 10 čepic za den, byla by výrobní kapacita 600 čepic za měsíc.
To by mělo pokrýt základní poptávku a umožnit nám růst v budoucnu.

\subsection{Dodavatelé a distribuce}

Pro zajištění kvalitní výroby čepic pro kočky bude firma Hats for Cats spolupracovat s ověřenými dodavateli materiálů a služeb.
Hlavní důraz bude kladen na kvalitu, udržitelnost a lokální spolupráci.

Pro textilie jsme se rozhodli zvolit lokální české dodavatele, kteří nabízí
kvalitní a hypoalergenní materiály, které jsou šetrné k pokožce zvířat a
neobsahují riziko alergických reakcí. Jedná se například o dodavatele VlnaHep\footnote{https://www.vlnahep.cz/}.
Jejich e-shop nabízí široký sortiment vlněných a bavlněných látek, které jsou vhodné pro výrobu čepic.
Cena jednoho klubka nitě bavlny(50g - 125m) se pohybuje kolem 46 Kč.

Dekorativní prvky, jako jsou mašličky, knoflíky nebo reflexní detaily, budou odebírány od specializovaných dodavatelů v Česku i EU.
Budou využity hlavně lokální obchody, nebo pokud bude mít zákazník specifický produkt, který by chtěl využít,
bude ho možné zakoupit i od specifického obchodu(s marží).

\vspace{20pt}
\noindent Pro distribuci bude firma využívat několik kanálů, které by měli pokrýt co nejširší spektrum zákazníků a tímpádem
zajistit vyšší tržby. Mezi tyto hlavní distribuční kanály patří:

\begin{enumerate}
  \item \textbf{E-shop} -
  Hlavním distribučním kanálem bude firemní e-shop dostupný na webových stránkách společnosti. 
  Tento kanál umožní přímý kontakt se zákazníky, efektivní správu objednávek a poskytne prostor pro personalizaci nabídky. 
  E-shop bude optimalizován pro mobilní zařízení a propojen s platformami sociálních sítí pro jednoduché sdílení produktů.
  \item \textbf{Online tržiště} -
  Pro zvýšení dosahu a získání nových zákazníků bude firma prodávat své produkty i na online tržištích jako je Etsy, Zboží.cz či Alegro.
  Tyto platformy umožní jednoduchou integraci s existujícími e-shopy a získání recenzí od zákazníků.
  Tímto bude mít značka možnost oslovit širší publikum a získat zpětnou vazbu na své produkty.
  \item \textbf{Podniková prodejna} -
  Vzhledem k tomu, že firma sídli v Brně a má zde svoji výrobu, bude mít možnost otevřít podnikovou prodejnu, kde bude možné si produkty prohlédnout, zakoupit osobně
  či si nechat vyrobit čepici na míru. Prodejna bude umístěna v blízkosti centra města, což zajišťuje snadný přístup pro zákazníky.
\end{enumerate}

Pro zasílání objednávek bude možnost užití služeb jako například DPD, PPL, či Česká pošta.
Jedná se o spolehlivé a (v zájmu možností) rychlé služby, které umožní doručení zásilek po celé České republice.

\pagebreak

\section{Charakteristika trhu}
\subsection{Konkurence}

Trh s produkty pro domácí mazlíčky, včetně specifických produktů jako čepice pro kočky, se v posledních letech rozvíjí a zájem o takové produkty stoupá.
Stoupající popularita domácích mazlíčků jako plnohodnotných členů rodiny vytváří prostor pro inovativní a stylové produkty, které kombinují funkčnost s estetikou.

\subsubsection{Přímá konkurence\cite{prima-konkurence}}
Na trhu existuje několik zahraničních i lokálních firem, které nabízejí obdobné produkty.
Mezi nejvýznamnější zahraniční konkurenty patří značky a trhy z USA a Japonska, které se specializují na designové produkty pro kočky.
Tyto značky však často cílí na globální trh a jejich produkty nejsou snadno dostupné v České republice, nebo jsou nabízeny za vysoké ceny
(díky clu, či VAT). Mezi takové trhy patří například Etsy, Amazon, nebo čínský AliExpress.

Na lokální úrovni je konkurence omezenější.
Existují menší řemeslné dílny a e-shopy, které nabízejí ručně vyráběné produkty pro kočky, ale tyto firmy se obvykle nespecializují přímo na kočky.
Poslední dobou se objevují e-shopy jako Allegro\footnote{https://allegro.cz/}(jedná se však pouze o velké tržiště - takový evropský AliExpress), nebo Kočičí páni\footnote{https://kocicipani.cz/}.
Navíc český trh v oblasti doplňků pro kočky není dostatečně pokryt, což představuje velkou příležitost pro značku naší firmy.

\subsubsection{Nepřímá konkurence\cite{neprima-konkurence}}
Nepřímou konkurenci představují firmy, které nabízejí obecné doplňky pro domácí mazlíčky, jako jsou obojky, oblečení nebo hřátky.
Tyto společnosti se zaměřují na širší portfolio produktů, ale čepice pro kočky nejsou jejich primárním zaměřením. Často se jedná o velké e-shopy nebo zverimexy, které se soustředí na masový prodej.

\subsubsection{Konkurenční výhoda}
Hats for Cats se od konkurence odlišuje svou jasnou specializací a originálním designem produktů. Naší prioritou je vysoká kvalita materiálů a pohodlí pro kočky,
což nám umožňuje nabídnout unikátní produkty, které nejsou na českém trhu běžně dostupné.
Navíc lokální výroba v Brně zaručuje rychlou dodávku a možnost personalizace podle přání zákazníků.

\vspace{10pt}
\noindent Díky omezené konkurenci a rostoucí poptávce po našich produktech pro kočky má naše firma silnou pozici pro vstup na trh.
Zaměření na lokální český trh s možností pozdější expanze na zahraniční trhy poskytuje společnosti dobrý základ pro dlouhodobý rozvoj.

\subsection{Zákazníci}

Do cílové kategorie naší společnosti patří především majitelé koček, kteří hledají stylové a kvalitní doplňky pro své mazlíčky.
Věková skupina zákazníků je široká, ale primárně se zaměřujeme na mladé a střední generace, které jsou aktivní na sociálních sítích a sledují nové trendy.
Skupiny bychom mohli rozdělit následovně:

\begin{enumerate}
  \item \textbf{Mladí majitelé koček} - tato skupina zákazníků je velmi aktivní 
  na sociálních sítích a hledá originální produkty pro kočky.
  \item \textbf{Módní nadšenci} - zákazníci, sledující nové trendy,
  případně se účastní módních akcí, jako jsou například výstavy koček.
  \item \textbf{Influenceři} - lidé, kteří mají vliv na ostatní uživatele 
  sociálních sítí a mohou propagovat naše produkty.
  \item \textbf{Zahraniční zákazníci} - zákazníci z jiných zemí, kteří hledají
  podobné produkty, ale nemají možnost je získat v jejich zemi. Tato skupina je pro
  nás velmi zajímavá pro budoucí expanzi, avšak se prozatím zaměřujeme na český trh.
\end{enumerate}

\pagebreak

\section{Marketingový plán}

\subsection{Cenová politika}

Naše cenová politika je postavená tak, aby reflektovala kvalitu našich produktů
a zároveň byla dostupná pro široké spektrum zákazníků.
Ceny jsou stanoveny na základě nákladů na výrobu, marže a cenové konkurence.

\vspace{10pt}

\noindent Výrobní cena jedné čepice se pohybuje okolo 30 - 50 Kč, v závislosti na použitých materiálech a složitosti designu.
Prodejní cena bude zahrnovat náklady na výrobu, marži a další náklady na provoz firmy.
Ve finančním plánu je zahrnuta cena fixních nákladů za měsíc 191 186 Kč, která zahrnuje mzdy zaměstnanců, nájem, energie a další náklady.
Pokud bychom vyrobili 600 čepic měsíčně, můžeme měsíční náklady vydělit počtem vyrobených čepic a získat marži na jednu čepici.
Tato marže by činila \textbf{318 Kč} na jednu čepici. Pro zajištění zisku a pokrytí dalších nákladů bychom měli stanovit prodejní cenu minimálně 400 Kč.
Dost také záleží na komplexitě designu a použitých materiálech, které mohou cenu zvýšit.
Orientační cenová politika by mohla vypadat následovně:

\renewcommand{\arraystretch}{1.2}

\begin{table}[h]
  \centering
  \begin{tabularx}{\textwidth}{ |>{\raggedright\arraybackslash}X||>{\raggedright\arraybackslash}X|X|>{\raggedright\arraybackslash}X| }
    \hline
    \textbf{Produktová řada} & \textbf{Popis} & \textbf{Cena} & \textbf{Příklad} \\
    \hline
    \hline
    \textbf{Základní řada} 
    & Jednoduché čepice z cenově dostupných materiálů. 
    & 300-700 Kč 
    & Klasická bavlněná čepice \\
    \hline
    \textbf{Prémiová řada} 
    & Stylové čepice z pokročilých materiálů (voděodolné, hypoalergenní, termoizolační). 
    & 800-1 200 Kč 
    & Voděodolná zimní čepice \\
    \hline
    \textbf{Limitované edice} 
    & Čepice inspirované aktuálními trendy. 
    & 1 000 Kč a více 
    & Vánoční čepice \\
    \hline
    \textbf{Speciální kolekce} 
    & Limitované edice z našeho automatu. 
    & 1 000 Kč a více 
    & Speciální letní čepice \\
    \hline
    \textbf{Balíčky} 
    & Zvýhodněné kombinace více produktů. 
    & 1 500-2 000 Kč 
    & 2 čepice + přenosná taška zdarma \\
    \hline
  \end{tabularx}
  \caption{Orientační přehled řad čepic a jejich cen}
\end{table}

\pagebreak

\subsection{Propagace}

V současné době, se k propagaci využívá
několik kanálů, které by měly oslovit cílovou skupinu a zvýšit povědomí o naší značce.

\begin{enumerate}
  \item \textbf{Sociální sítě} - V této době se jedná o velkou věc,
  dokáže oslovit velké množství lidí a zároveň je možné zde cílit na konkrétní skupiny(díky možnostem algoritmu).
  Mezi velké příklady patří například: Facebook, Instagram, TikTok, Pinterest.
  \item \textbf{SEO} - SEO(Search Engine Optimization), neboli optimalizace pro vyhledávače, je důležitým nástrojem pro zvýšení viditelnosti našeho e-shopu.
  Díky SEO můžeme získat vyšší pozice ve výsledcích vyhledávání a získat tak nové potenciální zákazníky.
  \item \textbf{PPC} - PPC(Pay-Per-Click) reklama je další možností, jak získat nové zákazníky.
  Jedná se o klasické placené reklamy, které se zobrazují na vyhledávačích nebo webových stránkách.
  \item \textbf{Influencer marketing} - Spolupráce s influencery je velmi efektivní způsob, jak oslovit specifickou cílovou skupinu.
\end{enumerate}

Ne všechny tyto kanály jsou však pro nás vhodné. Vzhledem k našemu omezenému rozpočtu a zaměření na lokální trh se zaměřujeme především na sociální sítě a SEO.

\subsubsection{SEO}

Jedná se o důležitý nástroj pro zvýšení viditelnosti našeho e-shopu ve vyhledávačích.
Společně se samotnými reklamami nám umožní být při hledání našich produktů mezi prvními výsledky.
Pro optimalizaci SEO je obvykle nutné najít klíčová slova, která naše cílovou skupina vyhledává.

\subsubsection{Reklamy na Instagramu}

K jedné z nejpoužívanějších sociálních sítí patří Instagram, který je velmi populární mezi mladými lidmi a módními nadšenci.
Pro naši firmu je Instagram ideálním místem pro propagaci našich produktů a získání nových zákazníků.
Na Instagramu lze umístit různé typy reklam, jako jsou obrázkové příspěvky, videa, Stories nebo Reels.
Reklamy se cení podle tzv. CPC(Cost-Per-Click), tedy za každého uživatele, který na reklamu klikne.

\section{Finanční plán}

\subsection{Počáteční investice a provozní náklady}

Naše firma zahajuje činost s počátečním kapitálem ve výši 530 000 Kč.
Tento kapitál by měl být dostatečný pro pokrytí základních nákladů,
jako jsou nájem, platy zaměstnanců, nákup materiálů a marketingové náklady.

\pagebreak

\subsubsection{DPH}

Zákon\footnote{https://www.zakonyprolidi.cz/cs/2004-235} uvádí,
že pokud obrat firmy přesáhne 2 000 000 Kč za posledních 12 měsíců,
musí se registrovat k dani z přidané hodnoty(DPH). Vzhledem k tomu, že
předpokládáme, že náš obrat bude větší než 2 000 000 Kč, budeme muset
být registrováni k DPH. DPH je 21\% z ceny produktu.
Pro výpočet DPH, které musíme státu odvést, musíme vzít 21\% z "hrubého příjmu"
a od něj odečíst 21\% z nákladu za materiál.
Tato částka bude muset být odvedena státu(každý 25. den v měsíci).
Pokud tedy vyděláme 360 000 Kč($600\cdot 600$) a materiál nás stál 27 600 Kč,
musíme odvést 69 804 Kč měsíčně.

\subsubsection{Daň z příjmu}

Zákon\footnote{https://www.zakonyprolidi.cz/cs/1992-586} o dani z příjmu uvádí, že
daň z příjmu se platí z čistého zisku firmy. Jedná se o sumu peněz, odečtenou
od uznatelných výdajů(zaměstnanci, materiál\dots) a nákladů. Samotnou daň
tvoří 19\% z čistého zisku a odvádí se jednou ročně. Pro příklad
můžeme uvést následující výpočet:
\begin{itemize}
  \item \textbf{Příjem} - 360 000 Kč
  \item \textbf{Výdaje} - 236 906 - 13 166 Kč = 223 740 Kč
  \item \textbf{Čistý zisk} - 136 260 Kč
  \item \textbf{Daň} - 19\% z 136 260 Kč = 25 889.4 Kč
\end{itemize}

\vspace{30pt}

\begin{table}[h]
  \centering
  \begin{tabular}{ | c | c | c | }
    \hline
    \textbf{Náklad} & \textbf{Cena} & \textbf{Počet} \\
    \hline \hline
    Stoly a židle & 50 000 Kč & 5 \\
    \hline
    Skříně a ostatní skladovací nábytek & 30 000 Kč & 3 \\
    \hline
    Šicí stroje & 250 000 Kč & 3 \\
    \hline
  \end{tabular}
  \caption{Počáteční náklady}
\end{table}

\noindent V počátečních nákladech nejsou zahrnuty věci jako PC, tiskárna nebo kancelářské potřeby, neboť
předpokládáme, že tyto věci budou již k dispozici zakladatelům.

\noindent Zde jsou dále uvedeny měsíční provozní náklady, které budou nutné pro chod firmy:

\begin{table}[h]
  \centering
  \begin{tabular}{ | c | c | }
    \hline
    \textbf{Náklad} & \textbf{Cena} \\
    \hline \hline
    Hrubá mzda & 24 000 Kč \\
    \hline
    Sociální poj. & 5 952 Kč \\
    \hline
    Zdravotní poj. & 2 160 Kč \\
    \hline
    \multicolumn{2}{|c|}{Celkem: 32 112 Kč} \tabularnewline % TODO: Recalculate
    \hline
  \end{tabular}
  \caption{Výdaje za zaměstnance}
\end{table}

% TODO: Pridat splaceni pujcky
\begin{table}[h]
  \centering
  \begin{tabular}{ | c | c | c | }
    \hline
    \textbf{Náklad} & \textbf{Cena} & \textbf{Typ} \\
    \hline \hline
    Nájem & 25 000 Kč & fixní \\
    \hline
    Náklady na zaměstnance & 32 112 Kč($\times$ 3) & fixní \\
    \hline
    DPH & 69 804 Kč & variabilní \\
    \hline
    Materiál & 27 600 Kč & variabilní \\
    \hline
    Marketing & 5 000 Kč & fixní \\
    \hline
    Splátka půjčky & 13 166 Kč & fixní \\
    \hline
    Daň z příjmu & 25 889 Kč & fixní \\
    \hline
    \multicolumn{3}{|c|}{Celkem: 262 795 Kč} \tabularnewline
    \hline
  \end{tabular}
  \caption{Měsíční provozní náklady}
\end{table}

\pagebreak

\subsubsection{Marketing a propagace}

Vzhledem k tomu, že předpokládáme, že naše firma bude obsahovat e-shop a
zakladatelé jsou již v zkušení v IT oboru, tak
počítáme s tím, že náklady na vytvoření e-shopu budou minimální - e-shop
bude vytvořen vlastními silami.

\vspace{10pt}
\noindent Co však bude vyžadovat investice, je samotný marketing a propagace.
Jak již bylo zmíněno, ze sociálních sítí se zaměříme především na Instagram a SEO.
SEO bude zařízeno námi, jako správci e-shopu(nebude tedy zapotřeba žádných výdajů). Co si budeme
muset zaplatit, budou reklamy na Instagramu(platformě Meta).

\noindent V následující tabulce jsou umístěny orientační ceny reklam na Instagramu:

\begin{table}[h]
  \centering
  \begin{tabular}{ |c|c| }
    \hline
    \textbf{Typ reklamy} & \textbf{Cena za kliknutí} \\
    \hline
    \hline
    Příspěvek ve feedu & 5-10 Kč \\
    \hline
    Stories & 3-7 Kč \\
    \hline
    Reels & 2-5 Kč \\
    \hline
    IGTV & 1-3 Kč \\
    \hline
  \end{tabular}
  \caption{Orientační ceny reklam na Instagramu \cite{reklamy-ig-zoomstudio}}
\end{table}

\noindent Pokud bychom si udělali jemný průzkum uživatelů Instagramu, tak bychom zjistili, že
naše cílová skupina převážně používá Stories a Reels, takže bychom se měli zaměřit na tyto typy reklam
a ideálně do nich investovat převážnou část našeho rozpočtu. Řekněme tedy, že optimální měsíční
rozpočet pro reklamy na Instagramu bude 5 000 Kč.

\subsection{Očekávané tržby}

Očekávané tržby jsou závislé na cenách našich produktů, výrobní kapacitě a poptávce na trhu.
I když je obtížné přesně odhadnout, kolik čepic budeme schopni prodat, můžeme vytvořit odhad na základě našich cen.
Průměrná cena, kterou náš zákazník utratí v našem e-shopu, by měla být kolem 600 až 700 Kč.
Měsíční výrobní kapacitou je 600 čepic, je však potřeba zohlednit, že ne všechny čepice budou prodány
(některé měsíce mohou být slabší než ostatní).

\subsection{Bod zvratu a návratnost investic}

Existuje obecná rovnice \cite{bod-zvratu}, která nám umožňuje spočítat bod zvratu, tedy bod, kdy se náklady vyrovnají příjmům.
Tento bod je důležitý pro určení, kdy se nám investice vrátí a kdy začneme generovat zisk a tímpádem
být schopni pokrýt další náklady a investice.

\begin{figure}[h!]
  \centering
  \captionsetup{justification=centering}
  \begin{equation}
    Q_{BZ}=\frac{F}{P-V}
  \end{equation}
  \caption{Rovnice bodu zvratu - $F$ je fixní náklad, $P$ je cena za jednotku, $V$ je variabilní náklad}
\end{figure}

\noindent Pokud bychom chtěli do rovnice zahrnout i zisk z jedné prodané jednotky, můžeme použít následující upravenou verzi rovnice:

\begin{figure}[h!]
  \centering
  \captionsetup{justification=centering}
  \begin{equation}
    Q_{BZ}=\frac{F+\text{zisk}}{P-V}
  \end{equation}
\end{figure}

\noindent Za využití těchto rovnic, bychom došli k závěru, že bod zvratu by měl být dosažen po vyrobění
a prodeji 546 čepic.

\section{Rizika}

V našem podnikatelském plánu existuje několik rizik, která by mohla ohrozit naše podnikání:

\begin{enumerate}
  \item \textbf{Nedostatečná poptávka} - Pokud by naše produkty nezaujaly zákazníky, mohli bychom mít problém s prodejem.
  \item \textbf{Konkurence} - Pokud by se na trhu objevila silná konkurence, mohli bychom ztratit zákazníky.
  Toto také zahrnuje možnost, že konkurence nabídne podobné produkty za nižší cenu.
  \item \textbf{Nedostatečná výrobní kapacita} - Pokud bychom nebyli schopni pokrýt poptávku, mohli bychom ztratit zákazníky.
\end{enumerate}

\section{Závěr}

Náš podnikatelský záměr pro firmu \textbf{Hats for Cats s.r.o.} je zaměřen na výrobu a prodej čepic pro kočky.
Společnost se bude specializovat na kvalitní a stylové produkty, které budou vyrobeny z materiálů a budou přizpůsobeny pohodlí koček.
Naše firma bude mít vlastní výrobu v Brně a bude nabízet produkty přes e-shop, online tržiště a podnikovou prodejnu.

\section{Seznam použité literatury}
\printbibliography[heading=none]
\section{Přílohy}

\begin{figure}[h]
  \centering
  \includegraphics[width=0.8\textwidth]{files/logo.png}
  \caption{Prototyp loga firmy Hats for Cats}
\end{figure}

\listoftables

\end{document}