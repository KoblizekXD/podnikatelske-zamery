\documentclass[12pt, a4paper]{article}

\usepackage[T1]{fontenc}
\usepackage[utf8]{inputenc}

\usepackage{blindtext}
\usepackage{titlesec}
\usepackage{array}
\usepackage{url}
\usepackage[czech]{babel}
\usepackage[backend=biber,style=authoryear]{biblatex}
\usepackage{csquotes}
\usepackage{amsmath}
\usepackage[bottom]{footmisc}
\usepackage{caption}
\usepackage{tabularx}
\addbibresource{./zdroje.bib}

\usepackage{graphicx}

\graphicspath{files/}

\usepackage{tocloft}
\renewcommand{\cftsecleader}{\cftdotfill{\cftdotsep}}

\begin{document}

\nocite{csob-podnikatelsky-plan}
\author{Jan Prokůpek, Miroslav Sklenář}
\title{\textbf{Podnikatelský záměr}\break Hats for Cats s.r.o.}
\date{}

\maketitle

\pagebreak

\tableofcontents

\pagebreak

\section{Shrnutí}

\subsection{Jméno a místo podnikání}
Jedná se o fiktivní firmu \textbf{Hats for Cats s.r.o.} se sídlem v Brně.

\subsection{Obchodní koncept}
Hats for Cats je originální značka specializující se na výrobu a prodej stylových, pohodlných a funkčních čepic pro kočky. 
Naším cílem je přinést do světa mazlíčků špetku originality a elegance, která potěší jak majitele, tak jejich kočičí společníky.

\subsection{Prodávaný produkt/služba}
Nabízíme ručně vyráběné čepice pro kočky v různých stylech a velikostech, 
které jsou přizpůsobeny pohodlí a bezpečnosti zvířete. 
Produkty zahrnují sezónní kolekce (zimní čepice, letní kloboučky) i tematické modely (vánoční, narozeninové, apod.),
či čepice kompletně na míru.

\subsection{Plněná potřeba na trhu}
V současnosti chybí na trhu esteticky atraktivní a funkční doplňky pro kočky, které by splňovaly požadavky majitelů na kvalitu a design. Hats for Cats vyplňuje tuto mezeru a oslovuje rostoucí komunitu milovníků koček, kteří hledají jedinečné produkty pro své mazlíčky.

\subsection{Konkurenční výhoda}
Naše konkurenční výhody spočívají v unikátním designu, použití kvalitních a hypoalergenních materiálů a důrazu na ruční výrobu. Dále se zaměřujeme na silný marketing na sociálních sítích, který osloví cílovou skupinu mladých a kreativních zákazníků.
Mezi existující alternativy patří levné a nekvalitní produkty z Číny, které nejsou přizpůsobeny potřebám zvířat a často mohou
způsobit zdravotní problémy kočkám, nebo tržiště jako Etsy.

\subsection{Ziskovost}
Očekáváme, že během prvního roku dosáhneme návratnosti investic díky 
nízkým provozním nákladům 
a atraktivní marži na našich produktech.

\subsection{Momentální situace}
Projekt je ve fázi příprav, zahrnující tvorbu prvních prototypů, 
testování na trhu a sestavování marketingové kampaně.

\subsection{Účel podnikatelského plánu}
Podnikatelský plán slouží k získání investice na zahájení výroby, 
marketingovou propagaci a rozšíření distribuční sítě.

\subsection{Potřeba investice}
Základní vklad na zahájení výroby, marketing, tvorbu webové stránky, materály,
pronájem prostor a další, bude činit cca 3 000 000 Kč.

\section{Představení společnosti}
Společnost \textbf{Hats for Cats s.r.o.} byla založena 2 fyzickými osobami,
které podnikají dle živnostenského oprávnění. Dle společenské smlouvy
každý z majitelů vlastní 50\% podíl na společnosti a má rovnocenný podíl v zisku.
Jednatelem společnosti je Jan Prokůpek.

\vspace{10pt}

\noindent Hats for Cats s.r.o. je společnost zaměřená na výrobu a prodej různých typů čepic pro kočky.
Nabízí ručně vyráběné produkty, které jsou uzpůsobeny několika velikostem, tak aby seděly na většinu koček.
Lze si také nechat vyrobit čepici na míru.

\pagebreak

\section{Podnikatelský projekt}

Hats for Cats se zaměřuje na výrobu, prodej a distribuci módních doplňků pro domácí mazlíčky, konkrétně kočky. 
Hlavní obory podnikání zahrnují:

\begin{enumerate}
  \item Výroba textilních doplňků. Jedná se o ekonomickou činost s kódem CZ-NACE 14.19.
  \item Maloobchod a e-commerce - kód CZ-NACE 47.91.
\end{enumerate}

\subsection{Vstupní předpoklady}

\subsubsection{Oprávnění k provozování podniku}
\begin{itemize}
  \item Získání potřebného kapitálu.
  \item Založení společnosti s ručením omezeným (s.r.o.):
  \begin{itemize}
    \item Zápis do obchodního rejstříku.
    \item Zaplacení správního poplatku.
    \item Získání IČO.
    \item Notářsky ověřený zápis společenské smlouvy mezi společníky.
  \end{itemize}
  \item{Registrace k daním} - pokud bude obrat vyšší než 2 mio. Kč, bude nutné se registrovat k dani z přidané hodnoty.
\end{itemize}

\subsubsection{Požadavky na zaměstnance}
Zaměstnanci by měli být schopni ručně šít a zpracovávat textilní materiály.

\subsubsection{Materiální a technické vybavení}
\begin{itemize}
  \item Prostory pro výrobu, skladování a administrativu.
  \item Šicí stroje, nůžky, střihové šablony, materiály a ostatní vybavení pro výrobu.
  \item Zázemí pro administrativu a marketing(počítač, tiskárna atd.).
\end{itemize}

\subsection{Organizačně-právní forma podnikání}
Podnik bude již od počátku fungovat jako společnost s ručením omezeným (s.r.o.).
Tato forma podnikání byla zvolena pro možnost oddělení osobního majetku majitelů od majetku společnosti.
Zároveň umožňuje získání investic od externích investorů.

\subsection{Stadium rozvoje podniku}
Společnost se nacházi ve fázi zahájení(start-up). Zde bude kladen důraz primárně na:
\begin{itemize}
  \item Budování značky a získání prvních zákazníků
  \item Nastavení výroby a dodavatelských řetězců
  \item Vytvoření e-shopu a marketingové strategie
\end{itemize}

\subsection{Majetkoprávní vztahy}

\begin{itemize}
  \item Hmotný majetek
  \begin{itemize}
    \item Prostory pro výrobu, skladování a administrativu.
    \item Šicí stroje, nůžky, střihové šablony, materiály a ostatní vybavení pro výrobu.
    \item Zázemí pro administrativu a marketing(počítač, tiskárna atd.).
  \end{itemize}
  \item Nehmotný majetek
  \begin{itemize}
    \item Značka Hats for Cats
    \item Webové stránky(doména, hosting), účty na sociálních sítích
    \item Know-how a zkušenosti zakladatelů
    \item Grafické návrhy produktů, design a marketingové materiály.
  \end{itemize}
  \item{Finanční majetek} - základní kapitál společnosti, zdroje na zahájení výroby, popř. jinak zprostředkované zdroje.
\end{itemize}

\pagebreak

\subsection{Organizace podniku}

Hats for Cats s.r.o. bude mít systém řízení, který zahrnuje různé úrovně odpovědnosti.
Na vrcholové úrovni bude jednatel společnosti, který bude zodpovědný za celkové řízení a strategii firmy.
Pod ním bude vedoucí výroby, který bude mít na starosti koordinaci výroby čepic a řízení zaměstnanců v této oblasti.
Vedoucí bude ovšem i normálním zaměstnancem, který se bude podílet na výrobě.
Administrativu a marketing bude zajišťovat jednatel společně se druhým zakladatelem společnosti.
Budou se starat o komunikaci s dodavateli, zákazníky, správu objednávek a marketingové kampaně.

\vspace{10pt}

\begin{figure}[h]
  \centering
  \includegraphics[width=0.8\textwidth]{files/obr1.png}
  \caption{Organizační struktura společnosti}
\end{figure}

Sídlo společnosti se nachází v Brně, konkrétně v kancelářské budově v blízkosti centra města, což usnadňuje přístup pro zaměstnance i zákazníky.
Prostor je vybavený moderními kancelářemi a výrobními dílnami pro výrobu čepic.
Orientační bod je blízkost hlavního nádraží, což usnadňuje dopravu do firmy.
K dispozici je parkování pro zaměstnance i návštěvníky, a to v areálu budovy, což zajišťuje pohodlný přístup.
\vspace{10pt}

Firma bude fungovat s malým administrativním zázemím pro správu objednávek, komunikaci s dodavateli a zákazníky.
Provozní doba firmy bude od pondělí do pátku, od 8:00 do 16:00 hodin.
V této době bude probíhat jak výroba, tak i administrativa.
Firma se zaměří na efektivní komunikaci a rychlé dodání produktů zákazníkům.

\pagebreak

\subsubsection{BOZP}

Naše firma se zaměřuje na ruční výrobu čepic pro kočky. 
Vzhledem k povaze práce je nutné zajistit bezpečné pracovní prostředí pro všechny zaměstnance. 
Hlavní rizika se týkají práce s textilními materiály, šicími stroji, ostrými nástroji a ergonomie při dlouhodobém sezení.
\vspace{10pt}

Následující rizika by měla být zohledněna v rámci BOZP(tento list není konečný, avšak obsahuje rizika s nejvyšším potenciálem poškození):

\begin{enumerate}
  \item Riziko poranění ostrými nástroji - nůžky, jehly, špendlíky.
  \item Riziko poranění šicími stroji - šití na strojích s rizikem šití prstů.
  \item Riziko úrazu elektrickým proudem - při používání elektrických strojů.
  \item Riziko vzniku požáru - při práci s hořlavými materiály.
\end{enumerate}

\vspace{10pt}

Pro předejití těmto rizikům je nutné zajistit pravidla BOZP, která budou dodržována všemi zaměstnanci.
Je nutné všechny zaměstnance seznámit s těmito pravidly a pravidelně je kontrolovat. Tento list
obsahuje základní opatření, která by měla být dodržována:

\begin{enumerate}
  \item Používání ochranných pomůcek - rukavice, brýle, roušky.
  \item Pravidelná údržba strojů a kontrola jejich bezpečnosti.
  \item Pravidelná kontrola elektrických zařízení.
  \item Důrazné proškolení zaměstnanců pro práci s elektrickými zařízeními.
  \item Umístění hasicích přístrojů a školení zaměstnanců v jejich používání.
  \item Zákaz používání otevřeného ohně(včetně kouření) v dílně.
\end{enumerate}

\pagebreak

\subsection{Popis vyráběného výrobku}

Hats for Cats se zaměřuje na výrobu čepic pro kočky. Čepice jsou vyráběny 
z kvalitních a hypoalergenních materiálů, které jsou přizpůsobeny pohodlí a bezpečnosti zvířete.
Produkty zahrnují sezónní kolekce (zimní čepice, letní kloboučky) i tematické modely (vánoční, narozeninové, apod.),
či čepice kompletně na míru. Lze se je objednat v různých velikostech, tak aby seděly na většinu koček.

\begin{figure}[h!]
  \centering
  \includegraphics[width=0.35\textwidth, height=0.3\textwidth]{files/kocka1.png}
  \hspace{0.5cm}
  \vspace{0.5cm}
  \includegraphics[width=0.35\textwidth, height=0.3\textwidth]{files/kocka2.png}
  \hspace{0.5cm}
  \includegraphics[width=0.35\textwidth, height=0.3\textwidth]{files/kocka3.png}
  \caption{Obrázky koček s čepicemi}
\end{figure}

\section{Realizace}
\subsection{Výroba}

Jak již bylo zmíňeno, výroba se odehrává v prostorách firmy v Brně. 
\vspace{10pt}

Výroba čepic začíná výběrem kvalitních materiálů, které jsou jemné, elastické a bezpečné pro kočičí pokožku.
Používáme především přírodní látky jako bavlnu a vlnu, které jsou prodyšné a hypoalergenní.
Každý kus je ručně střižen, šit a dekorován našimi zkušenými zaměstnanci, kteří mají dlouholeté zkušenosti s textilní výrobou.

Výrobní proces obecně zahrnuje následující kroky:

\begin{enumerate}
  \item Výběr materiálů: Vybíráme látky, které jsou měkké a šetrné k pokožce koček.
  \item Střih a příprava: Látky jsou ručně stříhány podle předem navržených vzorců.
  \item Šití: Čepice jsou šity na šicích strojích
  \item Dekorace a dokončení: Pokud má čepice dekorace, jsou přidány ručně.
\end{enumerate}

Mezi materiály, které používáme, patří bavlna, jemná vlna a elastické tkaniny, které umožňují snadné nasazení a pohodlné nošení.

\subsubsection{Výrobní kapacita}

Výrobní kapacita firmy je závislá na počtu zaměstnanců, dostupných prostorách a poptávce.
Pro začátek se můžeme řídit následující rovnicí pro přibližný výpočet výrobní kapacity za měsíc:

\begin{equation}
  \text{Výrobní kapacita} = \text{zaměstnanci} \cdot \text{čepic za den} \cdot 20
\end{equation}

Pokud bysme měli 3 zaměstnance(1 vedoucí výroby a 2 šičky) a každá šička by mohla vyrobit 10 čepic za den, byla by výrobní kapacita 600 čepic za měsíc.
To by mělo pokrýt základní poptávku a umožnit nám růst v budoucnu.

\subsection{Dodavatelé a distribuce}

Pro zajištění kvalitní výroby čepic pro kočky bude firma Hats for Cats spolupracovat s ověřenými dodavateli materiálů a služeb.
Hlavní důraz bude kladen na kvalitu, udržitelnost a lokální spolupráci.

Pro textilie jsme se rozodli zvolit lokální české dodavatele, kteří nabízí
kvalitní a hypoalergenní materiály, které jsou šetrné k pokožce zvířat a
neobsahují riziko alergických reakcí. Jedná se například o firmu MoraviaTex\footnote{https://www.moraviatex.shop/},
která sídlí v okolí Brna a specializuje se na prodej textilních materiálů. Pokud
by však zákazník požadoval speciální materiál, který není dostupný u lokálních dodavatelů,
budou dostupné i zahraniční dodavatelé. Zde se však dodací lhůty mohou velice prodloužit.


Dodávky nití a dekorativních prvků budou zajištěny lokálními společnostmi
jako jsou Nitě Brno, která se specializuje na vysoce kvalitní švy a nitě s různými vlastnostmi, včetně pevnosti a odolnosti.
Dekorativní prvky, jako jsou mašličky, knoflíky nebo reflexní detaily, budou odebírány od specializovaných dodavatelů v Česku i EU.

\vspace{20pt}
\noindent Pro distribuci bude firma využívat několik kanálů, které by měli pokrýt co nejširší spektrum zákazníků a tímpádem
zajistit vyšší tržby. Mezi tyto hlavní distribuční kanály patří:

\begin{enumerate}
  \item \textbf{E-shop} -
  Hlavním distribučním kanálem bude firemní e-shop dostupný na webových stránkách společnosti. 
  Tento kanál umožní přímý kontakt se zákazníky, efektivní správu objednávek a poskytne prostor pro personalizaci nabídky. 
  E-shop bude optimalizován pro mobilní zařízení a propojen s platformami sociálních sítí pro jednoduché sdílení produktů.
  \item \textbf{Online tržiště} -
  Pro zvýšení dosahu a získání nových zákazníků bude firma prodávat své produkty i na online tržištích jako je Etsy, Zboží.cz či Alegro.
  Tyto platformy umožní jednoduchou integraci s existujícími e-shopy a získání recenzí od zákazníků.
  Tímto bude mít značka možnost oslovit širší publikum a získat zpětnou vazbu na své produkty.
  \item \textbf{Podniková prodejna} -
  Vzhledem k tomu, že firma sídli v Brně a má zde svoji výrobu, bude mít možnost otevřít podnikovou prodejnu, kde bude možné si produkty prohlédnout, zakoupit osobně
  či si nechat vyrobit čepici na míru. Prodejna bude umístěna v blízkosti centra města, což zajišťuje snadný přístup pro zákazníky.
\end{enumerate}

Pro zasílání objednávek bude možnost užití služeb jako například DPD, PPL, či Česká pošta.
Jedná se o spolehlivé a rychlé(v zájmu možností) služby, které umožní doručení zásilek po celé České republice.

\pagebreak

\section{Charakteristika trhu}
\subsection{Konkurence}

Trh s produkty pro domácí mazlíčky, včetně specifických produktů jako čepice pro kočky, se v posledních letech rozvíjí a zájem o takové produkty stoupá.
Stoupající popularita domácích mazlíčků jako plnohodnotných členů rodiny vytváří prostor pro inovativní a stylové produkty, které kombinují funkčnost s estetikou.

\subsubsection{Přímá konkurence\cite{prima-konkurence}}
Na trhu existuje několik zahraničních i lokálních firem, které nabízejí obdobné produkty.
Mezi nejvýznamnější zahraniční konkurenty patří značky a trhy z USA a Japonska, které se specializují na designové produkty pro kočky.
Tyto značky však často cílí na globální trh a jejich produkty nejsou snadno dostupné v České republice, nebo jsou nabízeny za vysoké ceny
(díky clu, či VAT). Mezi takové trhy patří například Etsy, Amazon, nebo čínský AliExpress.

Na lokální úrovni je konkurence omezenější.
Existují menší řemeslné dílny a e-shopy, které nabízejí ručně vyráběné produkty pro kočky, ale tyto firmy se obvykle nespecializují přímo na kočky.
Poslední dobou se objevují e-shopy jako Allegro\footnote{https://allegro.cz/}(jedná se však pouze o velké tržiště - takový evropský AliExpress), nebo Kočičí páni\footnote{https://kocicipani.cz/}.
Navíc český trh v oblasti doplňků pro kočky není dostatečně pokryt, což představuje velkou příležitost pro značku naší firmy.

\subsubsection{Nepřímá konkurence\cite{neprima-konkurence}}
Nepřímou konkurenci představují firmy, které nabízejí obecné doplňky pro domácí mazlíčky, jako jsou obojky, oblečení nebo hřátky.
Tyto společnosti se zaměřují na širší portfolio produktů, ale čepice pro kočky nejsou jejich primárním zaměřením. Často se jedná o velké e-shopy nebo zverimexy, které se soustředí na masový prodej.

\subsubsection{Konkurenční výhoda}
Hats for Cats se od konkurence odlišuje svou jasnou specializací a originálním designem produktů. Naší prioritou je vysoká kvalita materiálů a pohodlí pro kočky,
což nám umožňuje nabídnout unikátní produkty, které nejsou na českém trhu běžně dostupné.
Navíc lokální výroba v Brně zaručuje rychlou dodávku a možnost personalizace podle přání zákazníků.

\vspace{10pt}
\noindent Díky omezené konkurenci a rostoucí poptávce po našich produktech pro kočky má naše firma silnou pozici pro vstup na trh.
Zaměření na lokální český trh s možností pozdější expanze na zahraniční trhy poskytuje společnosti dobrý základ pro dlouhodobý rozvoj.

\subsection{Zákazníci}

Do cílové kategorie naší společnosti patří především majitelé koček, kteří hledají stylové a kvalitní doplňky pro své mazlíčky.
Věková skupina zákazníků je široká, ale primárně se zaměřujeme na mladé a střední generace, které jsou aktivní na sociálních sítích a sledují nové trendy.
Skupiny bychom mohli rozdělit následovně:

\begin{enumerate}
  \item \textbf{Mladí majitelé koček} - tato skupina zákazníků je velmi aktivní 
  na sociálních sítích a hledá originální produkty pro kočky.
  \item \textbf{Módní nadšenci} - zákazníci, sledující nové trendy,
  případně se účastní módních akcí, jako jsou například výstavy koček.
  \item \textbf{Influenceři} - lidé, kteří mají vliv na ostatní uživatele 
  sociálních sítí a mohou propagovat naše produkty.
  \item \textbf{Zahraniční zákazníci} - zákazníci z jiných zemí, kteří hledají
  podobné produkty, ale nemají možnost je získat v jejich zemi. Tato skupina je pro
  nás velmi zajímavá pro budoucí expanzi, avšak se prozatím zaměřujeme na český trh.
\end{enumerate}

\pagebreak

\section{Marketingový plán}

\subsection{Cenová politika}

Naše cenová politika je postavená tak, aby reflektovala kvalitu našich produktů
a zároveň byla dostupná pro široké spektrum zákazníků.
Ceny jsou stanoveny na základě nákladů na výrobu, marže a cenové konkurence.

\renewcommand{\arraystretch}{1.2}

\begin{table}[h!]
  \centering
  \begin{tabularx}{\textwidth}{ |>{\raggedright\arraybackslash}X||>{\raggedright\arraybackslash}X|X|>{\raggedright\arraybackslash}X| }
    \hline
    \textbf{Produktová řada} & \textbf{Popis} & \textbf{Cena} & \textbf{Příklad} \\
    \hline
    \hline
    \textbf{Základní řada} 
    & Jednoduché čepice z cenově dostupných materiálů. 
    & 300-700 Kč 
    & Klasická bavlněná čepice \\
    \hline
    \textbf{Prémiová řada} 
    & Stylové čepice z pokročilých materiálů (voděodolné, hypoalergenní, termoizolační). 
    & 800-1 200 Kč 
    & Voděodolná zimní čepice \\
    \hline
    \textbf{Limitované edice} 
    & Čepice inspirované aktuálními trendy. 
    & 1 000 Kč a více 
    & Vánoční čepice \\
    \hline
    \textbf{Speciální kolekce} 
    & Limitované edice z našeho automatu. 
    & 1 000 Kč a více 
    & Speciální letní čepice \\
    \hline
    \textbf{Balíčky} 
    & Zvýhodněné kombinace více produktů. 
    & 1 500-2 000 Kč 
    & 2 čepice + přenosná taška zdarma \\
    \hline
  \end{tabularx}
  \caption{Přehled produktových řad čepic a jejich cen}
\end{table}


\subsection{Propagace}

\section{Finanční plán}
\section{Rizika}
\section{Závěr}
\section{Seznam použité literatury}
\printbibliography[heading=none]
\section{Přílohy}

\listoftables

\end{document}